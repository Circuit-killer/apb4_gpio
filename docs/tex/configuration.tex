\chapter{Configurations}\label{configurations}

\section{Introduction}\label{introduction-1}

The Roa Logic AHB-Lite APB4 GPIO is a fully configurable General Purpose
Input/Output core. The core parameters and configuration options are
described in this section.

\section{Core Parameters}\label{core-parameters}

\begin{longtable}[]{@{}lccl@{}}
\toprule
Parameter & Type & Default & Description\tabularnewline
\midrule
\endhead
\texttt{PDATA\_SIZE} & Integer & 8 & APB4 Data Bus \& GPIO Size\tabularnewline
\texttt{INPUT\_STAGES} & Integer & 3 & Number of \texttt{GPIO\_I} input synchronization
stages\tabularnewline
\bottomrule
\caption{Core Parameters}
\end{longtable}

\subsection{PDATA\_SIZE}\label{pdata_size}

The \texttt{PDATA\_SIZE} parameter specifies the width of the APB4 data bus and
corresponding GPIO interface width. This parameter must equal an integer
multiple of bytes.

\subsection{INPUT\_STAGES}\label{input_stages}

The APB4 GPIO inputs are sampled on the rising edge of the APB4 bus
clock (\texttt{PCLK}). As these inputs may be asynchronous to the bus clock, the
core automatically synchronises these signals and the \texttt{INPUT\_STAGES}
parameter determines the number of synchronisation stages.

Increasing this parameter reduces the possibility of metastability due
to input signals changing state while being sampled, but at the cost of
increased latency.

The minimum and default value of the \texttt{INPUT\_STAGES} parameter is 3

\subsection{Registers}\label{registers}

The APB4 GPIO core implements 4 user accessible registers. These are
described below:

\begin{longtable}[]{@{}lccl@{}}
\toprule
\textbf{Register} & \textbf{Address} & \textbf{Access} & \textbf{Function}\tabularnewline
\midrule
\endhead
\texttt{INPUT} & \texttt{Base + 0x3} & Read Only & Input Data Store\tabularnewline
\texttt{OUTPUT} & \texttt{Base + 0x2} & Read/Write & Output Data Store\tabularnewline
\texttt{DIRECTION} & \texttt{Base + 0x1} & Read/Write & Output Enable
control\tabularnewline
\texttt{MODE} & \texttt{Base + 0x0} & Read/Write & Push-Pull or Open-Drain
Mode\tabularnewline
\bottomrule
\caption{User Registers}
\end{longtable}

\subsection{INPUT}\label{input}

INPUT is a \texttt{PDATA\_SIZE} bits wide read-only register accessible at the
address 0x3.

On the rising edge of the APB4 Bus Clock (\texttt{PCLK}) input data on pins
\texttt{GPIO\_I} is sampled, synchronised and stored in the \texttt{INPUT} register where
it may be read via the APB4 Bus Interface.

\subsection{OUTPUT}\label{output}

OUTPUT is a \texttt{PDATA\_SIZE} bits wide read/write register accessible at the
address \texttt{0x2}.

Data to be transmitted from the APB4 GPIO core via the \texttt{GPIO\_O} \&
\texttt{GPIO\_OE} buses is written from the APB4 Interface to the OUTPUT
register.

\subsection{DIRECTION}\label{direction}

DIRECTION is a \texttt{PDATA\_SIZE} bits
wide active-high read/write register, accessible at the address \texttt{0x1}, and
controls the output enable bus \texttt{GPIO\_OE}.

\begin{longtable}[]{@{}cl@{}}
\toprule
\textbf{DIRECTION[n]} & \textbf{Direction}\tabularnewline
\midrule
\endhead
0 & Input\tabularnewline
1 & Output\tabularnewline
\bottomrule
\caption{DIRECTION Register}
\end{longtable}

\subsection{MODE}\label{mode}

MODE is a PDATA\_SIZE bits wide Read/Write register accessible at the
address 0x0. It individually sets the operating mode for each signal of
the GPIO\_O and GPIO\_OE buses as either push-pull or open drain, as
follows:

\begin{longtable}[]{@{}cl@{}}
\toprule
\textbf{MODE[n]} & \textbf{Operating Mode}\tabularnewline
\midrule
\endhead
0 & Push-Pull\tabularnewline
1 & Open Drain\tabularnewline
\bottomrule
\caption{MODE Register}
\end{longtable}

In push-pull mode, data written to the OUTPUT register directly drives
the output bus \texttt{GPIO\_O}. The \texttt{DIRECTION} register is then used to enable
\texttt{GPIO\_O} to drive the IO pad when set to `Output' mode (`1').

In open-drain mode, \texttt{GPIO\_O} is permanently driven low (`0') and the
OUTPUT register sets \texttt{GPIO\_OE} to `Output' to assert a logic `0' on the
bus, or `Input' (i.e. High-Z) to assert a logic `1'.

