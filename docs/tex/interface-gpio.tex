\section{GPIO Interface}\label{gpio-interface}

\begin{longtable}[]{@{}lccl@{}}
\toprule
Port & Size & Direction & Description\tabularnewline
\midrule
\endhead
\texttt{GPIO\_I} & \texttt{PDATA\_SIZE} & Input & Input Signals\tabularnewline
\texttt{GPIO\_O} & \texttt{PDATA\_SIZE} & Output & Output Signals\tabularnewline
\texttt{GPIO\_OE} & \texttt{PDATA\_SIZE} & Output & Output Enable Signal\tabularnewline
\bottomrule
\caption{GPIO Interface Signals}
\end{longtable}

\subsection{GPIO\_I}\label{gpio_i}

\texttt{GPIO\_I} is the input bus. The bus is \texttt{PDATA\_SIZE} bits wide and each bit
is sampled on the rising edge of the APB4 bus clock \texttt{PCLK}. As the inputs
may be asynchronous to the bus clock, synchronisation is implemented
within the core. 

\subsection{GPIO\_O}\label{gpio_o}

\texttt{GPIO\_O} is the output bus and is \texttt{PDATA\_SIZE} bits wide. Data is driven
onto the output bus on the rising edge of the APB4 bus clock \texttt{PCLK}.

\subsection{GPIO\_OE}\label{gpio_oe}

\texttt{GPIO\_OE} is an active-high Output Enable bus and is \texttt{PDATA\_SIZE} bits
wide.

The specific functionality of the \texttt{GPIO\_OE} bus is defined by the \texttt{MODE}
register. In push-pull mode it is used to enable a bidirectional output
buffer whose input is driven the \texttt{GPIO\_O} bus

In open-drain mode the \texttt{GPIO\_OE} bus is used to enable a logic `0' to be
driven from the \texttt{GPIO\_O} bus, and a logic '1' by disabling (`High-Z') the
output buffer.